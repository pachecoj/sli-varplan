\documentclass[twoside]{article}

\usepackage{amsfonts}
\usepackage{amsthm}
\usepackage{amsmath}
\usepackage{amssymb}
\usepackage{upgreek}
\usepackage[cm]{fullpage}
%%\usepackage{algorithmic}
%%\usepackage{algorithm} % must read after hyperref
\usepackage[colorlinks={true}, citecolor=blue, linkcolor=blue]{hyperref}       % hyperlinks
\usepackage{algpseudocode}
\usepackage{bbm}
\usepackage{bm}
\usepackage{graphicx}
\graphicspath{{./img/}}
\newtheorem{proposition}{Proposition}

%% \usepackage{array}
%% \usepackage{comment,array,wasysym}
%% \usepackage{graphicx,color,colortbl}

%% % \usepackage{mathptmx} % Use Times as default text font, and provide maths support.  I hate what it does to mathcal so I comment it out
%% % \usepackage{mathtools}
%% % \usepackage{multimedia}
%% % \usepackage{subfigure}
%% %\usepackage{ulem} % for strikethrough \sout
%% \usepackage[normalem]{ulem} % for strikethrough \sout, but avoid the annoying underline in \emph



%% % 
%% % \usepackage[utf8x]{inputenc}
%% % \usepackage{default}

%% % AISTATS sty file doesn't play nicely with caption and subcaption pkgs
%% % \usepackage{caption} 
%% %\usepackage{subcaption}

%% \usepackage{caption}
%% \DeclareCaptionType{copyrightbox}  % Without this, the AISTATs sty creates some clash
%% \usepackage{subcaption}


%% \usepackage{ifthen}
%% \usepackage{enumerate}


%% % 
%% % 
%% %         
%% % \usepackage{yhmath}  % I want really wide tilde. Oren Freifeld, 05/19/2013       




%% %% ICML PACKAGES %%
%% %% \usepackage{ellipsis}

%% %% \usepackage{multirow}

%% %% % Recommended, but optional, packages for figures and better typesetting:
%% %% \usepackage{microtype}
%% %% \usepackage{graphicx}
%% %% % \usepackage{subfigure}
%% %% \usepackage{booktabs} % for professional tables

%% %% % hyperref makes hyperlinks in the resulting PDF.
%% %% % If your build breaks (sometimes temporarily if a hyperlink spans a page)
%% %% % please comment out the following usepackage line and replace
%% %% % \usepackage{icml2018} with \usepackage[nohyperref]{icml2018} above.
%% %% \usepackage{hyperref}

%% %% % Attempt to make hyperref and algorithmic work together better:
%% %% \newcommand{\theHalgorithm}{\arabic{algorithm}}


%% %
%% % Commenting macros
%% %
%% \newcommand{\BLUE}[1]{{{\color{blue}{#1}}}}      
\newcommand{\RED}[1]{{{\color{red}{#1}}}}     
%% \newcommand{\MAGENTA}[1]{{{\color{magenta}{#1}}}}     
%% \definecolor{darkgreen}{rgb}{0,.5,0 }
%% \newcommand{\DARKGREEN}[1]{{{\color{darkgreen}{#1}}}}      
%% \newcommand{\TBD}{\RED{[--TBD--]}}
%% \newcommand{\TODO}[1]{\RED{[--TODO: #1--]}}
%% \newcommand{\BRAINDUMP}[1]{\DARKGREEN{[BRAIN DUMP: #1]}}
%  
%  \newcommand{\OREN}[1]{\RED{[Oren says: #1]}}
% % PICK YOUR COLOR...
\newcommand{\jwf}[1]{\BLUE{<\textbf{JWF}: #1 >}}
% \newcommand{\SUE}[1]{\MAGENTA{[Sue says: #1]}}

% MRF
\newcommand{\pot}{\psi}
\newcommand{\epmarg}{q}
\newcommand{\logpart}{\Phi}

% min / max
\DeclareMathOperator*{\argmax}{arg\,max\;}
\DeclareMathOperator*{\argmin}{arg\,min\;}

% Info Theory Stuff
\newcommand{\KL}[2]{\mathrm{KL}(#1\,\|\,#2)}

%
% List macros
%
\newcommand{\bi}{\begin{itemize}}
\newcommand{\ei}{\end{itemize}}
\newcommand{\deriv}{\mathrm{d}}

\newcommand{\etal}{\textit{et al}}
\newcommand{\FIG}{Fig.~}
%\newcommand{\FIGS}{Figs.~}

 \newcommand{\SEC}{Sec.~}

\newcommand{\eg}{\textit{e.g.}~}
\newcommand{\ie}{\textit{i.e.}~}
\newcommand{\cf}{\textit{c.f.}~}


\newcommand{\EQN}{Eqn.~}
% \newcommand{\EQN}{Equation }
\newcommand{\EQNS}{Eqns.~}
% \newcommand{\EQNS}{Equations }

% Expectation and Probability
\newcommand{\EE}{\ensuremath{\mathbb{E}}}
\newcommand{\PP}{\ensuremath{\mathbb{P}}}


% distributions
\newcommand{\Dir}{\ensuremath{\text{Dirichlet}}}
 
% integers
\newcommand{\ZZ}{\ensuremath{\mathbb{Z}}}
\newcommand{\RR}{\ensuremath{\mathbb{R}}}
\newcommand{\Rtwo}{\ensuremath{\RR^2}}
\newcommand{\Rthree}{\ensuremath{\RR^3}}
\newcommand{\Rn}{\ensuremath{\RR^n}}
% 
%positive integers
\newcommand{\Zplus}{\ensuremath{\ZZ^+}}
%positive reals
\newcommand{\Rplus}{\ensuremath{\RR^+}}


% n by n matrices
\newcommand{\nBynMats}{\ensuremath{\RR^{n \times n}}}
% m by n matrices
\newcommand{\mBynMats}{\ensuremath{\RR^{m \times n}}}
% n by p matrices
\newcommand{\nBypMats}{\ensuremath{\RR^{n \times p}}}
% 2 by 2
\newcommand{\TwoByTwoMats}{\ensuremath{\RR^{2 \times 2}}}
% 3 by 3
\newcommand{\ThreeByThreeMats}{\ensuremath{\RR^{3 \times 3}}}
% 2 by 3
\newcommand{\TwoByThreeMats}{\ensuremath{\RR^{2 \times 3}}}




% \newcommand{\EqualsDef}{\,{\overset{\text{def}}{=}}\,}
\newcommand{\EqualsDef}{\triangleq}


% set notation
% \newcommand{\set}[1]{\ensuremath{{\left\{#1\right\}}}}
\newcommand{\set}[1]{\ensuremath{{\{#1\}}}}


\newcommand{\InnerProduct}[2]{\left\langle #1,#2 \right\rangle}

\newcommand{\norm}[1]{{{\left\|#1\right\|}}}
\newcommand{\sign}[1]{{\mathrm{sign}\left(#1\right)}}


\newcommand{\ellTwoNorm}[1]{\norm{#1}_{\ellTwo}}


\newcommand{\Acal}{\mathcal{A}}
\newcommand{\Bcal}{\mathcal{B}}
\newcommand{\Ccal}{\mathcal{C}}
\newcommand{\Dcal}{\mathcal{D}}
\newcommand{\Ecal}{\mathcal{E}}
\newcommand{\Fcal}{\mathcal{F}}
\newcommand{\Gcal}{\mathcal{G}}
\newcommand{\Mcal}{\mathcal{M}}
\newcommand{\Ncal}{\mathcal{N}}
\newcommand{\Ocal}{\mathcal{O}}
\newcommand{\Pcal}{\mathcal{P}}
\newcommand{\Qcal}{\mathcal{Q}}
\newcommand{\Tcal}{\mathcal{T}}
\newcommand{\Ucal}{\mathcal{U}}
\newcommand{\Vcal}{\mathcal{V}}
\newcommand{\Wcal}{\mathcal{W}}
\newcommand{\Xcal}{\mathcal{X}}
\newcommand{\Ycal}{\mathcal{Y}}

\newcommand{\MATRIX}[2][cccccccccccccccccccc]{\left[
 \begin{array}{#1}
 #2
 \end{array}
\right]}


\newcommand{\TRACE}{\mathrm{trace}}

\newcommand{\var}[1]{\text{var} \big( #1 \big) }
\newcommand{\cov}[2]{\text{cov} \big( #1,#2 \big)}
\newcommand{\myt}[1]{\widetilde {#1} }

% bold font math 
\bmdefine\balpha{\alpha}
\bmdefine\bbeta{\beta}

\bmdefine\bphi{\phi}


\bmdefine\bb{b}

\bmdefine\bx{x}
\bmdefine\by{y}
\bmdefine\bdotx{\dot{x}}
\bmdefine\bxzero{x_{0}}
\bmdefine\bxone{x_{1}}

\bmdefine\bu{u}

\bmdefine\bv{v}
\bmdefine\br{r}


\bmdefine\bA{A}
\bmdefine\bB{B}

\bmdefine\bS{S}

\bmdefine\bU{U}

\bmdefine\bV{V}
\bmdefine\bX{X}
\bmdefine\bY{Y}


%\bmdefine\balpha{\alpha}

 \newcommand{\Nc}{{N_{c}}}
% \newcommand{\Nc}{N}
\newcommand{\Ne}{{N_{e}}}

% homo coo
\newcommand{\bxh}{\widetilde{\bx}}




\newcommand{\GLn}{\ensuremath{\mathrm{GL(n)}}}
\newcommand{\gln}{\ensuremath{\mathfrak{gl(n)}}}

\newcommand{\GLone}{\ensuremath{\mathrm{GL(1)}}}
\newcommand{\GLtwo}{\ensuremath{\mathrm{GL(2)}}}
\newcommand{\glTwo}{\ensuremath{\mathfrak{gl}(2)}}


% CPA Stuff

\newcommand{\Ac}[1]{A_{c(#1)}}
\newcommand{\Bc}[1]{B_{c(#1)}}

\newcommand{\Vaff}{\Vcal_{\mathrm{aff}}}

\newcommand{\Vpa}{\Vcal_{\mathrm{PA}}}
\newcommand{\Vcpa}{\Vcal_{\mathrm{CPA}}}


\newcommand{\Faff}{\Fcal_{\mathrm{aff}}}

\newcommand{\Fpa}{\Fcal_{\mathrm{PA}}}
\newcommand{\Fcpa}{\Fcal_{\mathrm{CPA}}}

\newcommand{\NSTEPS}{N_{\mathrm{STEPS}}}
\newcommand{\nsteps}{n_{\mathrm{steps}}}


\newcommand{\SigmaPA}{\Sigma_{\mathrm{PA}}}

\newcommand{\SigmaCPA}{\Sigma_{\mathrm{CPA}}}



\newcommand{\Vat}[1]{\bv{(#1)}}
\newcommand{\VatX}{\Vat{\bx}}
\newcommand{\Tat}[2]{T{(#1,#2)}}
\newcommand{\Tinvat}[2]{T^{-1}{(#1,#2)}}
\newcommand{\VatT}[1][t]{\Vat{\Tat{\bx}{#1}}}
% \newcommand{\VatT}[1][t]{\Vat{\Tat{\bx,#1}}}
\newcommand{\VatTinv}[1][t]{\Vat{\Tinvat{\bx}{#1}}}


\newcommand{\IntVatTofX}[1][t]{\int_{0}^{#1}\VatT[\tau]\, d\tau}

\newcommand{\AtimesX}{A{\bxh}}
 

\newcommand{\CpaVatX}{\Ac{\bx}\bxh}
% \newcommand{\CpaVatT}[1][t]{\Ac{\Tat{\bx}{#1}}\widetilde{T}(\bx,#1)}
\newcommand{\CpaVatT}[1][t]{\Ac{\Tat{\bx}{#1}}\widetilde{\Tat{\bx}{#1}}}

\newcommand{\IntCpaVatT}[1][t]{\int_{0}^{#1} \CpaVatT[\tau] \, d\tau}


 

\newcommand{\IntCpaVatTinv}[1][t]{\int_{0}^{#1}  \Ac{\Tinvat{\by}{#1}}\widetilde{\Tinvat{\by}{#1}} \, d\tau}


\newcommand{\CpaVatXasLinComb}{\sum_{j=1}^{d}\alpha_k\Bc{\bx}^{j}\bxh}
\newcommand{\CpaVatTasLinComb}[1][t]{\sum_{j}^{d}\alpha_j\Bc{\Tat{\bx}{#1}}^{j}\widetilde{\Tat{\bx}{#1}}}


\newcommand{\IntCpaVatTLinComb}[1][t]{\int_{0}^{#1}  \CpaVatTasLinComb[\tau] \, d\tau}


% USAGE:
% $\Vat{\bx}$
% $\VatX$
% $\Tat{\bx}{t}$
% $\VatT$
% $\VatT[\tau]$
% $\VatTinv[\tau]$
% $\IntVatTofX$
% $\AtimesX$
% $\Ac{\bx}$
% $\CpaVatX$
% $\IntCpaVatT$
% $\CpaVatXasLinComb$
% $\IntCpaVatTLinComb$



%\newcommand{\VEE}[1]{{#1}^{\vee}}
\newcommand{\VEE}[1]{\mathrm{vec}({#1})}

\newcommand{\HAT}[1]{{#1}^{\wedge}}

 \newcommand{\LCONSTRAINTS}{L_{\mathrm{constraints}}}

% Graph stuff
\newcommand{\parent}{\text{Pa}}
\newcommand{\interventions}{\mathcal{I}}
\newcommand{\alldata}{\mathcal{X}}

% Planning stuff
\newcommand{\actionset}{\mathcal{A}}


%\usepackage{aistats2019}
% If your paper is accepted, change the options for the package
% aistats2019 as follows:
%
\usepackage{aistats2019}
%
% This option will print headings for the title of your paper and
% headings for the authors names, plus a copyright note at the end of
% the first column of the first page.

% If you set papersize explicitly, activate the following three lines:
%\special{papersize = 8.5in, 11in}
%\setlength{\pdfpageheight}{11in}
%\setlength{\pdfpagewidth}{8.5in}

% If you use natbib package, activate the following three lines:
\usepackage[round]{natbib}
\renewcommand{\bibname}{References}
\renewcommand{\bibsection}{\subsubsection*{\bibname}}

% If you use BibTeX in apalike style, activate the following line:
%\bibliographystyle{apalike}

\begin{document}

% If your paper is accepted and the title of your paper is very long,
% the style will print as headings an error message. Use the following
% command to supply a shorter title of your paper so that it can be
% used as headings.
%
%\runningtitle{I use this title instead because the last one was very long}

% If your paper is accepted and the number of authors is large, the
% style will print as headings an error message. Use the following
% command to supply a shorter version of the authors names so that
% they can be used as headings (for example, use only the surnames)
%
%\runningauthor{Surname 1, Surname 2, Surname 3, ...., Surname n}

\twocolumn[

  \aistatstitle{Variational Information Planning: Efficient Methods for Approximate Decision Making}
]

%% \aistatsauthor{ Author 1 \And Author 2 \And  Author 3 }
%% \aistatsaddress{ Institution 1 \And  Institution 2 \And Institution 3 } ]

\begin{abstract}
  We consider the planning problem arising from active inference over
  a latent quantity.  Each stage poses a set of actions which must be
  evaluated for their potential reduction in posterior uncertainty.  A
  natural measure of uncertainty reduction is the mutual information
  (MI), which lacks a closed form for non-trivial models.  We propose
  an efficient planning mechanism which maximizes a variational lower
  bound on MI which, when combined with variational inference leads to
  efficient action selection and posterior updates.  We further
  characterize optima of the variational planning objective under
  exponential family approximations and show that optimal bounds arise
  from moment matching conditions under an augmented distribution.  We
  demonstrate our approach on three examples from the literature
  beginning with nonlinear target tracking in a fixed sensor network,
  gene regulatory network inference, and active learning in a
  semi-supervised labeled latent Dirichlet allocation (LLDA) model.
\end{abstract}

\section{Introduction}
Bayesian machine learning research has paid much attention to
developing posterior inference algorithms for models where posterior
calculation is computationally prohibitive.  Little focus, by
contrast, has been given to methods for decision making based on the
results of inference.  Such problems are of paramount importance in
reinforcement learning (RL), however the context and details of RL
planning differ substantially from our own setting.

In this paper we explore methods for sequential decision making, where
action choices drive data collection.  We further consider the setting
of \emph{information planning}, where decisions are generated by
maximizing the mutual information (MI) utility~\citep{WilliamsThesis}.
The setting most closely resembles Bayesian experiment
design~\citep{lindley56}, where experiments are chosen to minimize
uncertainty over a quantity of interest.  Indeed, MI has long been
used as a design utility since it is a measure of expected reduction
in posterior uncertainty~\citep{blackwell50, bernardo79a}.

Unlike experiment design, which typically assumes the cost of a
measurement dominates that of inference, our focus is on high
throughput sequential decision systems.  Where the former relies on
Monte Carlo inference and MI estimates during planning, we present a
comprehensive approach to inference and planning based on efficient
variational approximations.  

Our approach, which we call variational information planning (VIP),
maintains a series of variational approximations to the posterior and
MI utility.  For the planning stage, VIP extends a well-known lower
bound of MI~\citep{agakov2004algorithm} to the sequential setting.
During planning, the MI lower bound is optimized over an auxiliary
distribution, which approximates the posterior under a hypothesized
measurement.  We demonstrate that this optimization is convex for a
certain class of auxiliary models.  We establish optimality conditions
for the natural parameters of this family, and show that they are a
relaxation of the well known moment matching conditions.

%% Our setting differs from both Bayesian experimental design and RL in a
%% number of ways.  Traditional experiment design typically assumes the
%% cost of observation outweighs that of inference, and thus relies on
%% Markov chain Monte Carlo (MCMC) methods for inference.  MI is then
%% estimated over samples, resulting in estimator bias with slow
%% convergence~\citep{zheng2018robust, rainforth2018nesting}.  In
%% comparison to RL planning, our utility of interest is thus purely
%% exploratory, thus contrasting with the exploration-exploitation
%% trade-off common in RL planning~\citep{sutton1998reinforcement}.
%% Other differences with RL include the use of a structured statistical
%% model and representation of latent variables.

%% Statistical experiment design typically presumes that the cost of
%% experiments, or the cost of observation, far exceed the cost of
%% computation, and therefore rely on Markov chain Monte Carlo (MCMC)
%% methods for computation.  Such an approach is impractical for
%% high-throughput decision systems.  Moreover, Monte Carlo estimates of
%% MI have been shown to be biased due to the use of nested Monte Carlo
%% estimation, and that the rate of bias decay can be
%% slow~\citep{zheng2018robust, rainforth2018nesting}.

A core challenge we face is that variational approximations of
posterior uncertainty can be arbitrarily
poor~\citep{giordano2015linear, turner2011two}, despite their good
predictive accuracy.  Since MI is a measure of uncertainty, a naive
variational approximation will tend to yield poor decisions.  To
address these issues, our auxiliary distributions belong to a set
that, conditioned on a hypothesized observation, are in the
exponential family.  This set enables nonlinear dependence on the
conditioning variable, and is thus strictly larger than the set of
joint exponential family distributions.

In our experiments we demonstrate that VIP is sufficiently flexible to
apply in a variety of problem instances such as nonlinear target
tracking in a sensor network, experiment design, and active learning.
Moreover, VIP meets or exceeds the accuracy of methods based on exact
inference, MCMC requiring more computation, or specialized variational
approximations. 

%% and maximize a
%% well-known lower bound of MI resulting from Gibbs'
%% inequality~\citep{agakov2004algorithm}.  The bound is maximized over
%% an auxiliary model, which is strictly more expressive than the
%% exponential family posterior approximation, thus allowing for better
%% representations of uncertainty.  Moreover, we show that when this
%% class of models is conditionally in the exponential family, the
%% resulting optimization problem is convex.  We also show connections to
%% MI approximation based on moment matching operations, and that it is
%% equivalent to our approach in some restricted settings.  Finally, we
%% demonstrate effectiveness on a variety of problems including nonlinear
%% target tracking in a sensor network, gene regulatory network
%% inference, and active learning in the labeled LDA model (LLDA).




\section{Sequential Information Planning}
Consider a model of latent variables $x$ and observations $\Ycal_{t-1}
= \{y_1,\ldots,y_{t-1}\}$.  At each time $t-1$ a discrete
\emph{action} $a_{t-1} \in \{1,\ldots,A\}$ parameterizes the
likelihood, denoted \mbox{$p_{a_{t-1}}(y_{t-1} \mid x)$}.  Let
$\Dcal_{t-1} = \{\Ycal_{t-1},\Acal_{t-1}\}$ be the set of observations
and chosen actions \mbox{$\actionset_{t-1} =
\{a_1,\ldots,a_{t-1}\}$} at time $t-1$.  The posterior is then,
\begin{equation}\label{eq:conditional_indep_joint}
  p(x\mid \Dcal_{t-1}) \propto
    p(x) \prod_{i=1}^{t-1} p_{a_i}(y_i \mid x)
\end{equation}
The goal of sequential information planning is to choose the sequence
of actions $\Acal$ that minimize entropy of the
posterior~\eqref{eq:conditional_indep_joint}.  Specifically, at time
$t$, an action $a_t$ is selected to maximize the posterior mutual
information,
\begin{align}\label{eq:post_mi}
  a_t^{*} &= \argmax_a I(X;Y_t \mid \Dcal_{t-1}) \notag \\
          &= \argmax_a H(X\mid \Dcal_{t-1}) - H_a(X \mid Y_t, \Dcal_{t-1})
          %% &= \argmax_a H(X\mid \Dcal_{t-1}) + H_a(Y_t\mid \Dcal_{t-1})
          %% - H_a(X, Y_t \mid \Dcal_{t-1}).
\end{align}
Once an action is selected, new observations are drawn from the
appropriate likelihood model $y_t \sim p_{a_t}(\cdot \mid x)$ and the
posterior is updated.

%% We assume in \EQN\eqref{eq:info_plan_obj} that the marginal
%% entropy \mbox{$H(X\mid \Dcal)$} is invariant to future actions, and
%% thus can be ignored for planning.

Calculating the posterior MI in \EQN\eqref{eq:post_mi} is complicated
for two reasons.  First, the entropy terms involve expectations under
the posterior distribution~\eqref{eq:conditional_indep_joint}.
Second, calculating the conditional entropy $H(X\mid Y, \Dcal)$ requires
evaluation of the posterior predictive distribution $p(y\mid \Dcal)$
as in,
\[
  H(X\mid Y, \Dcal) = \EE\left[
    - \log \frac{p(x,y\mid \Dcal)}{ p(y\mid \Dcal) } \right],
\]
where we have dropped explicit indexing on time.  One approach is to
estimate this over samples \mbox{$\{y^i_t\} \sim
p_a(y\mid \Dcal_{t-1})$}.  The resulting empirical plug-in estimator
of MI is,
%% \begin{equation}\label{eq:emp_mi}
%%   \hat{I}_a = \frac{1}{N} \sum_{i=1}^N \log \frac{ p_a(y_t^i \mid x^i)
%%   }{ \frac{1}{M} \sum_{j=1}^M p_a(y_t^i \mid x^{ij}) }.
%% \end{equation}
\begin{equation}\label{eq:est_margent}
  \hat{I}_a = - \frac{1}{N} \sum_{i=1}^N \log \frac{
    p_a(y_t^i\mid x^i) }{\frac{1}{M}
        \sum_{j=1}^M p_a(y_t^i \mid x^{ij})}.
\end{equation}
Independent samples $\{x^{ij}\}_{j=1}^M \sim p(x \mid \Dcal_{t-1})$
are required for each action, and observation sample, to ensure
estimates are independent, thus increasing sample complexity.  While
the estimator~\eqref{eq:est_margent} is consistent, it is biased.
Moreover, bias is known to decay slowly~\citep{zheng2018robust,
rainforth2018nesting}.

%% %% \mbox{$I(X;Y_t \mid \Ycal_{t-1}, \Acal_{t-1})$}. 
%% %% \begin{align}
%% %%   a_t^{*} &= \argmax_a I_a(X;Y_t \mid \Ycal_{t-1}, \Acal_{t-1}) \\
%% %%     &= H(X\mid \Ycal_{t-1}, \Acal_{t-1}) - H_a(X, Y_t \mid
%% %%       \Ycal_{t-1}, \Acal_{t-1}). \notag
%% %% \end{align}
%% \begin{equation}
%%   a_t^{*} = \argmax_a H(X) + H_a(Y_t) - H_a(X, Y_t). \notag
%% \end{equation}




\section{Variational Information Planning}
\begin{figure*}[t]
  \centering
  \begin{tabular}{ccc}
    \hspace{-10mm}\includegraphics[width=0.37\textwidth]{inf_approx} &
    \hspace{-10mm}\includegraphics[width=0.37\textwidth]{augmented_dist} &
    \hspace{-10mm}\includegraphics[width=0.37\textwidth]{nonlinear_approx}
  \end{tabular}

  \caption{\small \textbf{Variational information planning steps.}
    \emph{Left:} Given observations $\Ycal$ the posterior is
    approximated with a tractable family $q(x) \approx p(x\mid
    \Ycal)$.  \emph{Center:} To consider a new observation $y$, a
    local approximation is formed $\hat{p}(x,y) = q(x) p(y \mid x)$
    using the forward model.  \emph{Right:} The auxiliary distribution
    $\omega(x\mid y)$ minimizes $\KL{\hat{p}(x\mid y)}{\omega(x\mid
      y)}$ to bound $I(X;Y)$.  When the auxiliary distribution is in
    the exponential family we achieve efficient updates, yet allow
    nonlinear dependence on the conditioning variable $y$, thus
    yielding tighter MI bounds.}

  \label{fig:approx}
\end{figure*}

Motivated by the challenges of sample-based MI estimation, we
introduce an efficient variational approach.  Beginning with a lower
bound on MI, we extend this to sequential decision making and
formulate the calculations for a model where observations are
conditionally independent.  Finally, we show how VIP can be applied to
a more complex model common in the related setting of active learning.

\subsection{Variational Information Bound}
For any valid conditional distribution ${\omega(x\mid y)}$, Gibbs' inequality
admits the following lower bound on MI:
\begin{equation}\label{eq:varmi}
  I(X;Y) \geq H(X) + \EE_p[ \log \omega(X\mid Y) ].
\end{equation}
This bound has been independently explored in various
contexts~\citep{agakov2004algorithm, mohamed2015variational,
  gao2016variational, chen2018learning}.  In the remainder of this
paper we refer to ${\omega(x\mid y)}$ as the \emph{auxiliary
  distribution}.  The dual planning problem maximizes the
bound~\eqref{eq:varmi} w.r.t. this auxiliary distribution.  Each stage
of planning requires the posterior mutual information, ${I(X,Y_t\mid
  \Dcal_{t-1})}$ bounded by,
\begin{equation}
  H(X\mid \Dcal_{t-1}) + \EE_p[ \log \omega(X\mid Y) \mid
    \Dcal_{t-1}].
  \label{eq:varmi_seq}
\end{equation}
Calculating \EQN\ref{eq:varmi_seq} involves expectations over the
posterior distribution ${p(x,y_t \mid \Dcal_{t-1})}$, thus efficient
sequential planning requires further approximations.  The procedure is
most easily understood for a simple model of conditionally independent
observations, which we now discuss before moving to more complicated
settings.

%% First, variational inference yields the posterior approximation
%% \mbox{$q(x)\approx p(x\mid \Dcal_{t-1})$}.  Then, we construct a local
%% approximation over the joint distribution for each hypothesized
%% action,
%% \begin{equation}
%%   \hat{p}_a(x,y_t) \approx p_a(x,y_t\mid \Dcal_{t-1})\quad \text{For}\;
%%   a=1,\ldots,A.
%%   \label{eq:local_approx}
%% \end{equation}
%% Finally, we perform planning by optimizing a lower bound with respect
%% to an auxiliary distribution \mbox{$\omega(x\mid y_t)$} and select the
%% maximizing action $a_t$ for the next time step.  See
%% \FIG\ref{fig:approx} for an illustration.  

\subsection{Conditionally Independent Observations}

Consider the model in \EQN\eqref{eq:conditional_indep_joint} where
observations $y_1,\ldots,y_t$ are independent, conditioned on $x$.
Given the variational approximation \mbox{$q(x) \approx p(x \mid
  \Dcal_{t-1})$} we form a local approximation of the distribution
over the future measurement at time $t$,
\begin{equation}\label{eq:local_approx}
  \hat{p}_{a}(x,y_t) \equiv q_{t-1}(x) p_{a}(y_t \mid x) \approx p_a(x,y_t\mid \Dcal_{t-1}).
\end{equation}
Here, $p_{a}(y_t \mid x)$ is the true likelihood under the
hypothesized action $a$.  We then bound the MI under
$\hat{p}(\cdot)$ as,
\begin{equation}\label{eq:varmi_approx}
  H_{\hat{p}}(X) + \max_{a, \,\omega}  \;\EE_{\hat{p}_a}\left[ \log \omega(X \mid Y_{t})
  \right].
\end{equation}
Under this model the marginal entropy $H(X)$ is constant during
planning and can be ignored.  The bound~\eqref{eq:varmi_approx} can be
evaluated in parallel for all actions $1,\ldots,A$.
\FIG\ref{fig:approx} illustrates the role of each approximation in a
single planning stage, and how the approximations relate to the target
distributions.  \EQN\eqref{eq:varmi_approx} bounds mutual information
under the local approximation $\hat{p}(\cdot)$.  The conditions
ensuring \EQN\eqref{eq:varmi_approx} is a reliable surrogate to
\EQN\eqref{eq:varmi_seq} are the same as those for variational
inference to be effective.

\subsection{Semi-Supervised Annotation Model}\label{sec:annotation}

We now consider a more complicated model consisting of semi-supervised
\emph{annotations} $\{y_n\}_{n=1}^N$, a fixed set of data $\{z_n\}_{n=1}^N$,
and latent quantities $x$.  The joint distribution is given by,
\[
  p(x,y,z) = p(x) \prod_{n=1}^N p(z_n, y_n \mid x).
\]
Variations of this model are common
in active learning contexts~\citep{settles2012active}.  For example,
$y_n$ may be a class label for data element $z_n$.  Each learning
stage selects the most informative annotation $y_{n^*}$ maximizing
posterior MI:
\begin{equation}
  n^{*} = \argmax_n I(X;Y_n \mid \Dcal_{t-1})
\end{equation}
%% Where $p(x,y_n \mid \Dcal_{t-1}) \propto $ \vspace*{-3mm}
%% \[
%%   p(x \mid \Dcal_{t-1} \setminus \{z_n\}) p(z_n \mid
%%     x ) p(y_n \mid z_n).\vspace*{-3mm}
%% \]
%% Here, $p(x \mid \Dcal_{t-1} \setminus \{z_n\})$ represents the
%% posterior distribution after removing $z_n$ from the data, which
%% intuitively avoids double counting data $z_n$.
%%
Here $\Dcal_{t-1}$ is the set of all data $z_1,\ldots,z_N$ and the
currently observed annotations.  To form a local approximation
$\hat{p}(\cdot)$ we assume a posterior approximation that is a product
of nonnegative normalizeable factors,
\begin{equation}
 q(x) \propto \prod_{n=1}^N \psi_n(x)
\end{equation}
In expectation propagation (EP) parlance, factors $\psi(x)$ can be
interpreted as messages in a factor graph.  Similarly, EP defines the
concept of a \emph{cavity distribution} $q^{\backslash n}(x) \propto
q(x) / \psi_n(x)$, which expresses the posterior approximation having
removed $z_n$.  Our local approximation is then analogous to the EP
\emph{augmented distribution},
\begin{equation}
  \hat{p}(x, y_n) \propto q^{\backslash n}(x) p(z_n, y_n \mid x).
\end{equation}
The MI lower bound is then identical to~\eqref{eq:varmi_approx}.  More
complicated models with nuisance variables that must be integrated out
for planning can be handled in a similar manner, with additional
marginalization.  We consider such a setting for the labeled LDA
active learning example in Sec.~\ref{sec:llda}.




\section{Optimization for Exponential Families}\label{sec:optim}
%% \begin{figure}[t]
%%   \centering
%%   \includegraphics[width=0.4\textwidth]{linear_approx}

%%   \vspace{-5mm}\caption{\small VIP optimizes a lower bound on MI
%%   w.r.t.~a distribution $\omega(x\mid y)$ approximating the
%%   conditional $\hat{p}(x\mid y)$. We use a linear Gaussian
%%   approximation in this case.  }  
%% \end{figure} 

Optimization of the bound~\eqref{eq:varmi_approx} with respect to the
auxiliary distribution $\omega(x\mid y)$ can be complicated in
general.  In this section we consider optimization for the class of
auxiliary distributions which are in the exponential family, when
conditioned on a hypothesized measurement.  This flexible family
allows for nonlinear dependence on the observation $y$ as illustrated
in \FIG\ref{fig:approx} (right).  We show that the resulting
optimization is convex in the exponential family natural parameters
and that optimality conditions yield a relaxation of the moment
matching property for exponential families.

%% We represent conditional exponential families with natural parameters
%% that are themselves functions of the conditioning variable $y$.  With
%% a slight abuse of terminology, we refer to this as a \emph{link
%% function}.  This rich family of distributions is strictly larger than
%% the set of joint distributions $\omega(x,y)$ in the exponential
%% family, and thus allows tighter achievable bounds.  We conclude by
%% characterizing the optimization of link function parameters.


\subsection{Optimizing the Auxiliary Distribution}

Consider the set of conditional distributions in the exponential
family having density,
\begin{equation}
  \omega_\theta(x \mid y) = h(x)\exp\left( \theta(y)^T \phi(x,y) - A(\theta(y)) \right),
\end{equation}
with natural parameters $\theta(y)$ a function of the conditioning
variable, sufficient statistics $\phi(x,y)$, base measure $h(x)$ and
log-partition function $A(\theta(y))$.  Optimizing the bound
in \EQN\eqref{eq:varmi_approx} is equivalent to minimizing the cross
entropy,
\begin{equation}\label{eq:crossent}
  \theta^{*}(y) = \argmin_{\theta} J(\theta) \equiv \EE_{\hat{p}}[ - \log \omega_{\theta}(x \mid y) ].
\end{equation}
Convexity of $J(\theta)$ can be established by explicit calculation of
the Hessian.  Alternatively, by adding a constant \mbox{$-H(\hat{p})$}
we have the following problem, which is equivalent to $J(\theta)$ up
to constant terms,
\begin{equation}\label{eq:dual}
  \theta^*(y) = \argmin_\theta \EE_{\hat{p}_y}\left[ \KL{\hat{p}_{x\mid y}}{\omega_\theta} \right]
\end{equation}
For brevity we have introduced the shorthand \mbox{$\hat{p}_{x\mid
y} \equiv
\hat{p}(x\mid y)$}.  For any realization $Y=y$ the KL term is convex in
$\theta(y)$, a well known property of the exponential
families~\citep{wainwright_jordan}.  \EQN\eqref{eq:dual} is then a
convex combination of convex functions, thus convexity holds.

The optimal parameter function $\theta^{*}(y)$ is given by the
stationary point condition,
\begin{equation}\label{eq:stationary_point}
  \EE_{\hat{p}_y}\left[ \EE_{\omega_{\theta^{*}}}[ \phi(x,y) \mid Y=y ] \right] = \EE_{\hat{p}}[\phi(x,y)].
\end{equation}
This is a weaker condition than the standard moment matching property
of exponential families, which typically minimizes KL.
Under~\eqref{eq:stationary_point} moments of $\omega(x\mid y)$ must
match \emph{in expectation} w.r.t.~the marginal distribution $p(y)$,
but need not be equal for any particular realization $Y=y$.

%% \begin{gather}
%%   \omega(X \mid Y = y; \theta) = \exp\left( \theta(y)^T \phi(X) -
%%     A(\theta(y)) \right) \\
%%   A(\theta(y)) = \log \int_{\Xcal \times \Ycal} \!\!\!\!\!\exp\left( \theta(y)^T \phi(x) \right)
%% dx dy
%% \end{gather}

\subsection{Parameter Function Optimization}

Stationary conditions~\eqref{eq:stationary_point} are in terms of a
function $\theta(y)$ which is assumed to be parametric.  Let $\eta$ be
parameters of the function, denoted $\theta_{\eta}(y)$.  Stationary
conditions in terms of parameters $\eta$ are then,
\begin{equation*}
  \EE_{p_y}\left[ \left( D_\eta \theta \right)^T \EE_{\omega_\eta}[\phi(x,y)]
    \right]
    \!= \EE_{p_y}\left[ \left( D_\eta \theta \right)^T \EE_{p_{x\mid
          y}}[ \phi(x,y) ] \right]
\end{equation*}
where $D_\eta \theta$ is the Jacobian matrix of partial derivatives.
If $\theta(y)$ is convex in the parameters $\eta$ then the
optimization \EQN\eqref{eq:dual} remains convex.

In principle, the map $\theta_{\eta}(y)$ can be any parametric
function, for example a neural network with parameters $\eta$.
Indeed, in related work~\cite{chen2018learning} optimize the MI
bound~\eqref{eq:varmi} w.r.t.~a neural network map for feature
selection tasks.  However, such an approach violates the convexity
properties above and leads to computation that is prohibitive for
sequential decision making tasks.

%% One approach is to
%% represent $\theta_{\eta}(y)$ as a neural network, with parameters
%% $\eta$.  In this setting, the Jacobian can be efficiently calculated
%% for any $y$ via backpropagation.



\section{Evaluating the MI Bound}
The previous section characterized natural parameters maximizing the
MI bound~\eqref{eq:varmi_approx} w.r.t.~the auxiliary distribution.
Planning, however, requires the value of this bound at its optimum.
For some models this evaluation is straightforward, but others require
estimation.  We begin with a discussion of computing the bound for
complex models.  We conclude with a class of models for which
evaluation is simple, and corresponds to the standard moment matching
property.

\subsection{Bound Estimation}

%% For simple natural parameter functions, such as linear $\theta(y) =
%% Ay$, the MI bound can often be optimized and evaluated in closed
%% form~\citep{agakov2004algorithm}.  We make extensive use of this
%% approximation for our experiments in \SEC\ref{sec:experiments} for
%% this purpose.  More complicated functions, however, require estimation
%% of the bound.

To simplify the discussion, we focus on the conditionally independent
model with PDF~\eqref{eq:conditional_indep_joint}.  Recall the local
approximation $\hat{p}(x,y) = q(x)p(y\mid x)$, where we drop explicit
time indexing for brevity.  The relevant term in the
bound~\eqref{eq:varmi_approx} is the conditional cross
entropy,
\[
  \EE_{\hat{p}}[ -\log \omega(x \mid y)
  ] \approx - \frac{1}{N} \sum_{i=1}^N \EE_{\hat{p}_{y \mid
    x^i}}[ \log \omega(x^i \mid y) ]
\]
where samples $\{x^i\}_{i=1}^N \sim q(x)$.  Since $q(x)$ is a
tractable distribution, this step can be done efficiently.

The expectation $\EE_{y\mid x^i}[\cdot]$ is with respect to the
forward model (likelihood), and can often be computed in closed-form.
For some models, however, this term must be approximated, and requires
simulation of the forward model.  This step is also efficient,
assuming a Bayesian network, but leads to a higher variance estimate.
Both estimators are consistent by the LLN.

%% A similar approach can
%% be taken for estimating gradients, if closed-form solutions are not
%% available.

\subsection{Moment Matching Solution}\label{sec:moment_match}

%% The reader may consider an alternative approach to approximating MI,
%% which is as follows.  First, select a joint distribution $\omega(x,y)$
%% in the exponential family,
%% \[
%%   \omega_{\eta}(x,y) = h(x,y)\exp\left\{ \eta^T \phi(x,y) - A(\eta)
%%     \right\}.
%% \]
%% Next, approximate the distribution $\hat{p}(x,y)$ by minimizing
%% $\KL{\hat{p}}{\omega}$ via moment matching, $\EE_{\hat{p}}[ \phi(x,y)
%% ] = \EE_{\omega_{\eta}}[ \phi(x,y) ]$.  Finally, approximate mutual
%% information as $I_{\hat{p}}(X;Y) \approx I_{\omega}(X;Y)$.

%% We show how this approach is equivalent to the optimization
%% of \SEC\ref{sec:optim} in some cases.

Under some conditions the value of the MI
bound~\eqref{eq:varmi_approx} takes a simple form at its optimum.  To
see this, we first establish that standard moment matching of the
auxiliary distribution is optimal for some models.  We then show that
the value of the bound at this moment matching solution is equivalent
to calculating entropy under the auxiliary distribution, which is
tractable.

One class of models for which the bound is easily calculated are those
where the marginal $\hat{p}(y)$ is in the exponential family, for
example if $y$ is a discrete label.  Then, consider the following
joint exponential family,
\[
  \omega_{\eta}(x,y) = h(x,y)\exp\left\{ \eta^T \phi(x,y) - A(\eta)
    \right\}.
\]
Furthermore, consider the parameters $\eta^*$ satisfying the moment
matching property,
\begin{equation}\label{eq:moment_match}
  \EE_{\hat{p}}[ \phi(x,y) ] = \EE_{\omega_{\eta}}[ \phi(x,y) ].
\end{equation}
Moment matching, combined with the assumption that $\hat{p}(y)$ is in
the exponential family, implies that the marginal can be exactly
calculated $\omega_{\eta}(y) = \hat{p}(y)$.  Using this equivalence,
and rewriting~\eqref{eq:moment_match}, we have:
\begin{equation}\label{eq:moment_match_cond}
  \EE_{\hat{p}}[ \phi(x,y) ] = \EE_{\hat{p}_y}[ \EE_{\omega_{x\mid y}}[ \phi(x,y)
      \mid Y=y ] ],
\end{equation}
where $\omega_{\eta^{*}}(x\mid y)
= \omega_{\eta^{*}}(x,y) \div \int \omega_{\eta^{*}}(x,y) \deriv
x$.  \EQN\eqref{eq:moment_match_cond} is the optimality
condition~\eqref{eq:stationary_point} of the MI lower bound.  This
establishes that standard moment matching is optimal for the class of
models where $\hat{p}(y)$ is in the exponential family.

We now establish that the moment matching solution $\eta^{*}$ leads to
a simple form of the bound~\eqref{eq:varmi_approx}.  By direct
calculation, the cross entropy $H_{\hat{p}}(\omega_{\eta^*}(x,y))$
equals,
\begin{equation}
  \EE_{\hat{p}}[ - \log h(x,y) ] - \eta^T \EE_{\omega_{\eta^*}}[ \phi(x,y) ] + A(\eta).
\end{equation}
For distributions with constant base measure $h(x,y)$ we have that,
$H_{\hat{p}}(\omega_{\eta^*}(x,y)) = H(\omega_{\eta^*}(x,y))$.  By
similar logic for the marginal entropy, and by applying the entropy
chain rule, we have that:
\begin{equation}
  H_{\hat{p}}( \omega_{\eta^*}(x\mid y) ) = H(\omega_{\eta^*}(x,y)) - H(\omega_{\eta^*}(y)).
\end{equation}
The l.h.s.~is the relevant conditional entropy term from the MI
bound~\eqref{eq:varmi_approx}.  The r.h.s.~is the entropy of the joint
and marginal distributions $\omega(\cdot)$ at the optimal parameters,
which is closed form.  We have thus shown that the aforementioned MI
approximation is equivalent to optimizing the variational lower bound.


%% \subsection{BACKUP}
%% Consider a pair of joint and marginal distributions in the exponential
%% family,
%% \begin{gather*}
%%   \omega_{\eta}(x,y) = h(x,y)\exp\left\{ \eta^T \phi(x,y) - A(\eta)
%%     \right\}\\ % \equiv \omega_{xy}\\
%%   \omega_{\beta}(y) = h(y)\exp\left\{ \beta^T \phi(y) - A(\beta) \right\}.
%%   %\equiv \omega_y.
%% \end{gather*}
%% When $\omega(y) = \int \omega(x,y) \,\deriv x$ are marginally consistent,
%% we have that the conditional is $\omega(x\mid y) = \omega(x,y) \div
%% \omega(y)$.  As a result, the objective $J(\theta)$ can be
%% re-expressed as,
%% \begin{align}\label{eq:constrained_mibound}
%%   &\min_{\eta, \beta} \;J(\eta,\beta) \equiv \EE_p[ \log \omega_{\beta}(Y) ] - \EE_p[ \log
%%     \omega_{\eta}(X,Y) ] \notag \\
%%   &\;\text{s.t.} \; \EE_{\omega_{\beta}}[ \phi(Y) ] =
%%     \EE_{\omega_{\eta}}[ \phi(Y) ].
%% \end{align}
%% We have assumed here that $\int \omega(x,y) \,\deriv x$ is in the same
%% exponential family as $\omega(y)$.  Then, marginal consistency
%% is equivalent to the moment constraints above.  This assumption does not hold in
%% general but will simplify later discussion.

%% When the marginalization constraints are satisfied we have that
%% $J(\eta,\beta) = J(\theta)$ by construction, where $\theta$ can be
%% expressed in terms of the parameters $\eta$ and $\beta$.  The
%% problem~\eqref{eq:constrained_mibound} is then convex on the
%% constraint set, though not strictly so since many joint and marginal
%% distributions map to the same conditional.  

%% The objective $J(\eta,\beta)$ is not convex off of the constraints.
%% By adding constants $\EE_p[ \log p(X,Y) ]$ and $\EE_p[ - \log p(Y) ]$
%% we have the equivalent expression,
%% \begin{equation*}
%%   J(\eta,\beta) = \text{const.} + \KL{ p_{XY} }{ \omega_{\eta} } - \KL{ p_Y }{
%%     \omega_{\beta} },
%% \end{equation*}
%% which is convex in $\eta$ and concave in $\beta$ by convexity
%% properties of Kullback-Leibler for exponential families.

%% The zero gradient of $J(\eta,\beta)$ yields the moment matching
%% equations, 
%% \begin{align}
%%   \EE_p[ \phi(X,Y) ] &= \EE_{\omega_{\eta}}[ \phi(X,Y)
%%     ] \label{eq:statcond_joint} \\
%%   \EE_p[ \phi(Y) ] &= \EE_{\omega_{\beta}}[ \phi(Y) ]. \label{eq:statcond_marg}
%% \end{align}
%% A solution $\{\eta,\beta\}$ to the above equations is a feasible
%% point, given our assumption that $\int \omega(x,y) \,\deriv x$ and
%% $\omega(y)$ belong to the same exponential family.

%% Consider a model where the marginal $p(y)$ is in the exponential
%% family, for example when $p(x,y)$ is a mixture distribution with
%% discrete label $y$.  In this case, the moment matching condition above
%% means $\omega(y) = p(y)$ and thus,
%% \begin{equation}
%%   \EE_p[ \phi(X,Y) ] = \EE_{p_Y}[ \EE_{\omega_{X\mid Y}}[ \phi(X,y)
%%       \mid Y=y ] ],
%% \end{equation}
%% which is the solution of the unconstrained problem in
%% \SEC\ref{sec:optim}.  As a result, it is also a solution to the
%% constrained problem~\eqref{eq:constrained_mibound}, since the
%% objectives are equal on the constraint set by construction.

%% Given the moment matched solution we have that the cross entropy
%% equals, 
%% \[
%%   \EE_p[ - \log \omega_{\eta} ] = \EE_p[ - \log h(x) ] - \eta^T \EE_p[
%%     \phi(X,Y) ] + A(\eta).
%% \]
%% For distributions with constant base measure $h(x)$ the above equals
%% entropy of $\omega(x,y)$, since $\EE_p[ \phi(X,Y) ] = \EE_\omega[
%%   \phi(X,Y) ]$.  The same holds for the marginal entropy and, thus,
%% the conditional cross entropy, \mbox{$H_p( \omega(X\mid Y) ) = H_\omega(X
%% \mid Y)$}.





\section{Experimental Results}\label{sec:experiments}
We demonstrate VIP in a variety of contexts including sensor
selection, Bayesian experiment design, and active learning.  The
primary comparison in most settings is to MCMC inference or exact
numerical inference when possible, along with empirical mean
estimation of the MI for planning.  In this way, our motivation is to
demonstrate comparable accuracy using more efficient variational
methods.  For the more complex case of LLDA we in fact observe
sustained improvements over baseline.

\subsection{Sensor Selection}

We begin with estimating target position in a network of sensors, each
with fixed position.  Due to communication constraints we can draw
measurements from only a single sensor at each time.  At each planning
stage we must draw measurements from the most informative sensor.

We consider estimation in, both, static and dynamic settings.  In both
cases we optimize the MI lower bound over a linear Gaussian auxiliary
distribution $\omega(x \mid y_t) = \Ncal(a y_t, \sigma^2)$, which can be
solved in closed-form.  We compare the impact of inference on
predictive accuracy by comparing exact numerical calculation,
variational inference, and MCMC.


%% We model sensor
%% noise as depending on the relative distance to the unknown target;
%% sensors further from the target produce less accurate detections.  

\textbf{Static Estimation.}  A stationary target has position drawn
from a Gaussian prior $x \sim \Ncal(m,\sigma^2)$.  Observations are
drawn from one of $K$ sensors, each with fixed position $l_k$.  Sensor
noise is modeled as a two-component Gaussian mixture model,
\[
  y \mid x; k \sim w * \Ncal(0, v_0) + (1-w) * \Ncal(x, v_k(x)).
\]
The mixture consists of a noise distribution with fixed variance $v_0$
and an observation model with noise variance increasing with relative
distance: \mbox{$v_k(x) = |l_k - x| + v_1$}.

\begin{figure}
  \centering
  \begin{tabular}{cc}
    \hspace{-3mm}\includegraphics[width=0.24\textwidth]{ss_stateerr_maxIters100_Nsamp50_Nruns20.pdf} &
    \hspace{-5mm}\includegraphics[width=0.24\textwidth]{ss_MI_maxIters100_Nsamp50_Nruns20.pdf}
  \end{tabular}
  
  \caption{\small\textbf{Static target estimation.} Estimation of a 1D
    target position in a network of $K=10$ equally spaced sensors.
    Mean (solid) plus STDEV (dashed) over 20 realizations.  \emph{Left:}
    Variational planning based on EP inference yields comparable error
    in state estimate compared to exact inference with empirical MI
    estimates. ADF inference yields higher error.  \emph{Right:}
    Empirical estimates based on the numerical posterior most
    accurately estimate MI.  VIP bound gap is consistently lower for
    more accurate EP posterior estimates as compared to
    ADF.}
  \label{fig:static}
\end{figure}

\begin{figure*}[!t]
  \centering
  \hspace{-3mm}
  \includegraphics[width=1.0\textwidth]{tracking_single}

  \caption{\small\textbf{Nonlinear tracking example} in a field of
    $K=10$ equally spaced, stationary, sensors.  The best comparison
    is to numerical approximation to MI and the posterior distribution
    (\emph{top-center}).  For reference, we have also included an
    \emph{oracle} which selects the sensor closest to the true target
    location (\emph{left column}).  In typical cases such as this one,
    we see that VIP state error is comparable to empirical estimation
    under the same posterior approximation.  However, VIP shows lower
    accuracy when planning is computed against an approximate
    posterior, in this case particle filtering (PF).}

  \label{fig:tracking_single}
\end{figure*}

\textbf{Dynamical System.} We extend the setting to a nonlinear
dynamical system frequently used in the sequential Monte Carlo
literature~\citep{kitagawa1996monte, gordon1993novel,
  cappe2007overview},
\[
  \Ncal\left(0.5 x_{t-1} + 25 x_{t-1} / (1+x_{t-1}^2)
  + 8 \cos(1.2t), \sigma_u^2\right).
\]
To keep consistent with prior work we model observations
as, \mbox{$y_t \mid x_t; k \sim \Ncal(x_t^2/20, v_{tk}(x_t))$}.  The
variance function $v_{tk}(x_t)$ is identical to our static example.
To demonstrate flexibility additionally replace variational inference
with a particle filter.  \FIG\ref{fig:tracking_single} demonstrates an
example scenario.

%\subsubsection{Results}

\textbf{State predictions competitive with empirical.}  In both cases
VIP produces state estimates with similar or better accuracy to
empirical planning, depending on the chosen posterior approximation.
In the static estimation model we find similar accuracy between exact
numerical inference with empirical planning and EP inference with VIP
planning (\FIG\ref{fig:static}; left).  In the tracking model we also
consider exact inference with VIP planning, for which median error is
lowest.  When comparing particle filter inference VIP and empirical
planning accuracy are comparable, with the former showing slightly
lower median error (\FIG\ref{fig:dynamic}; left).

\textbf{Planning is sensitive to posterior accuracy.}  We also find
that estimates of the state estimate, and of the MI calculation, are
sensitive to accuracy of the posterior approximation.  In the static
case we compare against assumed density filtering (ADF), which is more
numerically stable than EP but tends to produce less accurate
posterior approximations.  Planning based on the ADF posterior
produces less accurate state estimates (\FIG\ref{fig:static}; left)
and higher error MI estimates (\FIG\ref{fig:static}; right).
Similarly, use of particle filter inference in the tracking model
increases error for both the state (\FIG\ref{fig:dynamic}; left) and
MI calculation (\FIG\ref{fig:dynamic}; right).



\begin{figure}[t]
  \centering
  \begin{tabular}{cc}
    \hspace{-3mm}\includegraphics[width=0.24\textwidth]{tracking_state_err} &
    \hspace{-3mm}\includegraphics[width=0.24\textwidth]{tracking_sensor_err}
  \end{tabular}
  
  \caption{\small\textbf{Nonlinear tracking for 20 random
  trials.}  \emph{Left:} Exact inference with variational planning
  yields the lowest RMS state error.  Particle filter inference with
  500 particles and variational planning (PFVI) yields lower median
  error compared to MC estimates of information (MCMI), though wider
  error quantiles.  \emph{Right:} Again, Exact VI shows the lowest RMS
  error of the selected sensor position w.r.t.~optimal, whereas PFVI
  and MCMI both have higher error, again with PFVI having larger
  quantiles.}
  \vspace{-4mm}
  \label{fig:dynamic}

\end{figure}





\subsection{Gene Regulatory Network}

\cite{steinke2007experimental}, and later~\cite{seeger2008bayesian},
introduced a heuristic approach to sequential experiment design which
shares components of our current method.  Their approach relies on EP
inference and related approximations of the MI utility.  Using an
example from these papers, we compare VIP for the discovery of causal
gene interactions.

Let $x\in\RR^n$ represent the deviation of $n$ gene expression levels
from steady state.  The matrix $A\in\RR^{n\times n}$ represents causal
interactions, with sparse entries drawn i.i.d.~from a Laplace
distribution,
\begin{equation}\label{eq:sparselin}
  \!\!\!p(A,x) \propto \prod_{i=1}^n \Ncal(u_i \mid a_i^T
    x, \sigma^2) \prod_{j=1}^n \textrm{Laplace}(a_{ij} \mid \lambda) 
\end{equation}
where $a_i$ is the $i^{\text{th}}$ row of matrix $A$.  Here $u$
represents an external control vector (perturbation).  Interventions
include up regulating $u_i>0$, down regulation $u_i<0$ and no
intervention $u_i=0$ for the $i^{\text{th}}$ gene.  Observe that the
joint~\eqref{eq:sparselin} is unnormalized since the likelihood term
is not a normalized distribution of $x$.

\begin{figure}[t]
  \centering
  \begin{tabular}{cc}
    \hspace{-3mm}\includegraphics[width=0.24\textwidth]{sparselin_iAUC_N100_norandom_run20_norepeats} &
    \hspace{-3mm}\includegraphics[width=0.22\textwidth]{sparselin_noise_N100_norandom_run20_norepeats}
  \end{tabular}
  \caption{\small\textbf{Sparse linear model.} Average +/- STDEV
    computed for $n=50$ nodes over 20 random networks.  \emph{Left:}
    AUC for edges with true weights $|a_{ij}|>0.1$ plotted for each
    intervention experiment.  VIP shows similar results to Steinke and
    Seeger with slight improvement for less than 25 experiments.
    \emph{Right:} Average AUC at varying noise levels are broadly
    similar to Steinke and Seeger. The dip at moderate SNR levels may
    be due to our choice of a simple approximating family (linear
    Gaussian).}
  
  \label{fig:sparselin}
\end{figure}

We compare to the variational method of~\cite{steinke2007experimental}
and later~\cite{seeger2008bayesian} which maintain a mean field
Gaussian posterior approximation using EP:
\[
q(A) = \prod_i p_i^{(0)}(a_i) \prod_j \widetilde{t}_{ij}(a_{ij}).
\]
Here, $p^{(0)}_i(a_i)$ approximates the base measure \mbox{$N(u_i \mid
  a_i^T x, \sigma^2)$} and \mbox{$\widetilde{t}_{ij}(a_{ij})$} the
Laplace factors.  Given past observations $\Dcal$ and a new control
and observation pair $\{x_*,u_*\}$, maximizing MI is (approximately)
equivalent to maximizing \mbox{$\EE_{x_*}\!\left[\, \KL{q'}{q}
    \,\right]$},
%% \begin{equation}\label{eq:max_kl}
%%   \max_{u_*} \; \EE_{x_*}\!\left[\, \KL{q'}{q} \,\right],
%% \end{equation}
where approximation comes from the update posterior \mbox{$q'(A)
  \approx p(A \mid \Dcal \cup \{x_*,u_*\})$}.  The previous authors
approximate expectation over samples $\{x_{*}^{k}\}$ from the
predictive distribution $p(x_{*} \mid \Dcal, u_{*})$.  Since this
approach requires updating the EP posterior $q'(A)$ for each sample
$x_{*}^k$, which is prohibitive, they instead propose a
non-iterative approximation which only updates the Gaussian base measure
$p^{(0)}(A)$ at each sample.

\textbf{Similar results to specialized method.} We optimize the MI
bound over a linear Gaussian approximation for 20 random trials and
report area under the curve (AUC) for edge prediction
(\FIG\ref{fig:sparselin}; left).  Steinke and Seeger approximate MI by
updating only the base measure $p^{(0)}$, which involves a moment
matching projection $\EE_{q'}[ \phi(A) ] = \EE_{\hat{p}}[ \phi(A) ]$,
where $\hat{p}(A)$ is the augmented distribution, which is similar to
the moment matching solution discussed in \SEC\ref{sec:moment_match}.
As a result, VIP reduces to a solution similar to that of Steinke and
Seeger, and results are comparable for more than 30 interventions.
Results remain similar across varying noise levels
(\FIG\ref{fig:sparselin}; right), albeit with a slight drop in
accuracy at moderate SNR.  We hypothesize that this intermediate
region depends more strongly on good MI approximations, and that VIP
would benefit from a more flexible approximating distribution than the
linear Gaussian one chosen.


\textbf{Improved AUC for fewer interventions.}  Despite similarity in
the methods, we do observe small improvements at early interventions.
In particular, Steinke and Seeger observe that their proposed
experimental design approach performs poorly for few interventions and
propose a hybrid method which randomly selects the first 20
interventions, then performs information guided selection thereafter.
Random selection still performs well in this regime.

%% which matches
%% moments of $q^{'}$ to those of.  Specifically, the
%% augmented distribution is given by,
%% \[
%%   \hat{p}(a_i, x_*) \propto \Ncal(u_* \mid a_i^T
%%   x_*, \sigma^2) \prod_j \widetilde{t}_{ij}(a_{ij}).
%% \]
%% The EP update then matches moments to the augmented distribution above
%% for each row $a_i$.
%% %% $\EE_{\hat{p}}[a_i] = \EE_{q'}[ a_i ]$ and $\text{var}_{\hat{p}}[ a_i
%% %% ] = \text{var}_{q'}[ a_i ]$ for each row $a_i$.
%% %% \[
%% %%   q' \Leftrightarrow \argmin_{q'} \KL{}{q'}
%% %% \]

\subsection{Active Learning for Labeled LDA}\label{sec:llda}

\begin{figure}[t]
  \centering
  \includegraphics[width=0.45\textwidth]{llda/llda_model.pdf}
  
  \caption{\small\textbf{Labeled LDA} graphical model augments
    standard LDA with semi-supervised annotations $y_{dn}$ shown as
    shaded dashed nodes.  Each stage of active learning selects
    unlabeled item $(d,n)$ maximizing mutual information
    $I(\Psi,Y_{dn})$ between topics $\Psi$ and annotations $Y$, as in
    \EQN\eqref{eq:maxmi_llda}.  }
  \label{fig:llda_model}
\end{figure}


\begin{figure*}
  \begin{tabular}{ccc}
    \hspace{-3mm}\includegraphics[width=0.32\textwidth]{llda/topics_random_runid_5} &
    \includegraphics[width=0.32\textwidth]{llda/topics_gibbs_runid_5} &
    \includegraphics[width=0.32\textwidth]{llda/topics_vip_runid_5} \\
    {\small\textbf{Gibbs + Random}} & {\small\textbf{Gibbs +
        Empirical}} & {\small\textbf{EP + VIP}}
  \end{tabular}

  \caption{\small \textbf{Learned LLDA topics} from a corpus of $D=50$
    documents, each with $N_d=25$ words drawn from the bars topics
    with a $W=25$ word vocabulary.  We model Topic 0 as a \emph{rare}
    topic (see text).  Gibbs estimates are averaged over $1000$
    samples drawn from parallel chains.  Topic estimates under EP
    inference with selection using VIP (\emph{right}) are broadly
    similar to Gibbs when using empirical MI estimates for selection
    (\emph{center}), though at reduced computational cost.  Gibbs
    estimates have higher noise in low probability regions.
    Annotation based on random selection (\emph{left}) performs poorly
    regardless of the inference method -- Gibbs shown.}

  \label{fig:llda_topics}
\end{figure*}

\begin{figure}[t]
  \centering
  \hspace{-4mm}\includegraphics[width=0.53\textwidth]{llda/llda_tverr} 
  
  \caption{\small\textbf{Total variation (TV) error} for all topics on
    the ``bars'' dataset across 10 random trials.  Variational
    planning with fully parameterized softmax shows consistent
    improvement over Gibbs on average (solid).  Gibbs estimates show
    tighter standard deviation (shaded) for planning based on MI
    estimates.  Both inference methods perform similarly for random
    selection.}
  
  \label{fig:llda_tv}
\end{figure}

%% \begin{figure}[t]
%%   \centering
%%   \includegraphics[width=0.45\textwidth]{llda/bound_comparison}
  
%%   \caption{\small\textbf{LLDA Bound Comparison.} Variational MI
%%   estimates for annotations using three approximating families, ranked
%%   in order of increasing value by empirical estimate of MI.  An ideal
%%   approximation would show monotonically increasing values with little
%%   gap.  The conditional Dirichlet model is a poor approximation.  Two
%%   variations on the softmax distribution provide tighter bounds, with
%%   the fully parameterized softmax being slightly
%%   preferred.}  \label{fig:llda_bound}
%% \end{figure}

Labeled LDA (LLDA) is a semi-supervised extension of
LDA~\citep{blei2003latent}.  For $K$ topics, $D$ documents
$d=1,\ldots,D$, and $N_d$ words per document, the standard
unsupervised LDA model is given by,
\begin{align*}
  \theta_d &\sim \text{Dirichlet}(\alpha) && \text{Topic proportions}\\
  \psi_k &\sim \text{Dirichlet}(\beta_k) && \text{Topics}\\
  z_{dn} \mid \theta_d &\sim \text{Cat}(\theta_d) && \text{Topic Labels}\\
  w_{dn} \mid z_{dn}, \psi &\sim \text{Cat}(\psi_{z_{dn}}) && \text{Words}
\end{align*}
As shown in \FIG\ref{fig:llda_model}, LLDA introduces semi-supervised
annotations for each word~\citep{flaherty2005latent}.  We model
annotations as noisy observations of the true topic assignment,
\[
  y_{dn} \mid z_{dn} \sim \text{Cat}(\pi_{z_{dn}}).
\]
This choice ensures interpretable topics as the posterior concentrates
on a preferred ordering of labels.  Other supervised LDA variations
exist~\citep{ramage2009labeled} but we do not consider them here.

Our evaluation uses the ``bars'' dataset~\citep{griffiths2004finding}
where topics can be visualized as a set of vertical and horizontal
\emph{bars} (see \FIG\ref{fig:llda_topics}).  To amplify the effect of
informative annotations we model a \emph{rare} topic with a
non-symmetric Dirichlet prior over topic proportions: $\alpha = (0.05, 1, 1,
\ldots, 1)^T$.


\textbf{Active Learning} At each learning stage $t$ the planner
selects an annotation $y_{d^*n^*}$ maximizing posterior MI:
\begin{equation}
  (d^*,n^*) = \argmax_{(d,n)} I(\Psi, Y_{dn} \mid \Wcal, \Ycal_{t-1}),
  \label{eq:maxmi_llda}
\end{equation}
with observed annotations $\Ycal_{t-1}$ and $\Wcal$ the corpus of
words.  \EQN\eqref{eq:maxmi_llda} is an instance of the annotation
model (\SEC\ref{sec:annotation}) as integration over nuisance
variables induces dependence among annotations.  We choose a softmax
auxiliary distribution:
\[
  \omega(y = k \mid \psi) \propto \exp( w_k^T \text{vec}(\psi_k) + w_{0k} ),
\]
which has a number of beneficial properties: first, it ensures
convexity of the planning objective~\eqref{eq:maxmi_llda} in the
parameters $w$ (\SEC\ref{sec:optim}).  Second, the marginal
distribution $p(y)$ is in the exponential family (discrete) thus
ensuring simple evaluation of the MI bound (\SEC\ref{sec:eval}).

\textbf{Lower topic error compared to MCMC.}  We compare VIP using EP
inference~\citep{broderick2013streaming} to empirical MI planning
based on Gibbs samples.  Observe that MI estimates require averaging
over topic samples, and thus must account for topic label switching.
We found that any reasonable alignment (e.g. MAP) resulted in poor
estimates~\citep{stephens2000dealing}.  We instead relabel Gibbs
samples to minimize absolute error w.r.t.~the \emph{true topics},
which would not be feasible in practice.  \FIG\ref{fig:llda_tv}
reports total variation (TV) error across topics $\sum_k \|\psi_k -
\hat{\psi}_k\|_1$, based on the posterior mean estimate $\hat{\psi}$.
We found that increasing from the reported 1000 Gibbs samples did not
lead to significant improvements.  We instead suspect that poor MCMC
results are due to a combination of MI estimator bias and local
optima.



%% MI as $I(\Psi,Y_{dn}) \geq H(Y_{dn})
%% + \EE[ \log \omega(Y_{dn} \mid \Psi) ]$ using the following softmax
%% distribution,
%% \begin{equation}
%%   \omega(Y = k \mid \Psi) \propto \exp( w_k^T \text{vec}(\Psi) +
%%   w_{0k} ).
%% \end{equation}

%% which is in the exponential family with natural parameters
%% $\theta(\Psi) = (w_1^T \text{vec}(\Psi) + w_{01}, \ldots,
%% w_K^T \text{vec} + w_{0K})^T$.







\section{Conclusion}
We have introduced an approach to sequential decision making (VIP)
which leverages the efficiency of variational approximations to
efficiently produce high quality decisions.  VIP avoids the pitfalls
associated with uncertainty estimates of variational methods, which
can be quite poor, by decoupling the posterior approximation from the
auxiliary model used in planning.  We have shown that the optimization
arising from conditionally exponential family auxiliary models is
convex in the natural parameters.  For simple auxiliary models this
convexity often translates into closed-form solutions, resulting in
high throughput decisions.  Using the same basic methodology we have
shown that VIP can be easily adapted to a variety of contexts, and
performs comparably to methods requiring more computation, such as
MCMC, or to specialized methods in the case of regulatory network
inference.


\bibliography{refs}
\bibliographystyle{abbrvnat}

\end{document}
