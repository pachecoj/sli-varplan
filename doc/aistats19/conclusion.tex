We have introduced VIP, an approach to sequential decision making
which leverages the efficiency of variational approximations to
produce high quality decisions at timescales that would be infeasible
with existing sample-based methods.  Using the same basic methodology
we have shown that VIP can be easily adapted to a variety of contexts,
and performs comparably to sample-based approaches requiring more
computation and to methods specialized for a particular problem, as in
the regulatory network example.  We look forward to future
applications of VIP in domains where decisions are time-sensitive.


%% By optimizing a lower bound on MI
%% over an auxiliary model, VIP avoids the pitfalls associated with
%% uncertainty estimates of variational methods.  We have shown that the
%% optimization arising from conditionally exponential family auxiliary
%% models is convex in the natural parameters.  Using the same basic
%% methodology we have shown that VIP can be easily adapted to a variety
%% of contexts, and performs comparably to methods requiring more
%% computation, such as MCMC, or to specialized methods in the case of
%% regulatory network inference.
