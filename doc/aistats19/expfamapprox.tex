The previous section characterized natural parameters maximizing the
MI bound~\eqref{eq:varmi_approx}.  Planning, however, requires
evaluation of the optimum.  For some models this is straightforward,
but others require estimation.  We begin with a discussion of
computing the bound for complex models.  We conclude with a class of
models for which evaluation is simple, and corresponds to the
standard moment matching property.

\subsection{Bound Estimation}

%% For simple natural parameter functions, such as linear $\theta(y) =
%% Ay$, the MI bound can often be optimized and evaluated in closed
%% form~\citep{agakov2004algorithm}.  We make extensive use of this
%% approximation for our experiments in \SEC\ref{sec:experiments} for
%% this purpose.  More complicated functions, however, require estimation
%% of the bound.

Let us focus on the conditionally independent model with
PDF~\eqref{eq:conditional_indep_joint}.  Recall the local
approximation $\hat{p}(x,y) = q(x)p(y\mid x)$, where we drop explicit
time indexing for brevity.  The relevant term
in the bound~\eqref{eq:varmi_approx} is the negative conditional cross
entropy,
\begin{equation}
  \EE_{\hat{p}}[ \log \omega(X \mid Y)
  ] \approx \frac{1}{N} \sum_{i=1}^N \EE_{\hat{p}_{Y \mid
    x^i}}[ \log \omega(x^i \mid Y) ]
\end{equation}
where samples $\{x^i\}_{i=1}^N \sim q(x)$ are drawn from the posterior
approximation, which is efficient by assumption.

The expectation $\EE_{Y\mid x^i}[\cdot]$ can be problematic, in which
case we simulate the forward model and perform standard Monte Carlo
integration.  This is also efficient given a Bayesian network, albeit
with higher variance.  Both estimators are consistent by the LLN.

%% A similar approach can
%% be taken for estimating gradients, if closed-form solutions are not
%% available.

\subsection{Moment Matching Solution}\label{sec:moment_match}

The reader may consider an alternative approach to approximating MI,
which is as follows.  First, select a joint distribution $\omega(x,y)$
in the exponential family,
\[
  \omega_{\eta}(x,y) = h(x,y)\exp\left\{ \eta^T \phi(x,y) - A(\eta)
    \right\}.
\]
Next, approximate the distribution $\hat{p}(x,y)$ by minimizing
$\KL{\hat{p}}{\omega}$ via moment matching, \mbox{$\EE_{\hat{p}}[ \phi(X,Y) ]
= \EE_{\omega_{\eta}}[ \phi(X,Y) ]$}.  Then, approximate
mutual information as $I_{\hat{p}}(X;Y) \approx I_{\omega}(X;Y)$.

We show how this approach is equivalent to the optimization
of \SEC\ref{sec:optim} in some cases.  Consider a model where the
marginal $p(y)$ is in the exponential family, for example when $y$ is
a discrete label.  The moment matching above then implies $\omega(y) =
p(y)$ and so,
\begin{equation}\label{eq:moment_match_cond}
  \EE_p[ \phi(X,Y) ] = \EE_{p_Y}[ \EE_{\omega_{X\mid Y}}[ \phi(X,y)
      \mid Y=y ] ],
\end{equation}
This shows that the moment matching solution solves the optimization
in \SEC\ref{sec:optim} for the
conditional \mbox{$\omega_{\theta}(x\mid y)
= \omega_{\eta}(x,y) \div \int \omega_{\eta}(x,y) \deriv x$}.

The moment matching solution leads to a simple bound evaluation.  To
see this, let $\eta^*$ be the parameters of $\omega_{\eta}(x,y)$
satisfying the moment matching conditions.  The cross entropy
$H_{\hat{p}}(\omega(x,y))$ then equals,
\begin{equation}
  \EE_{\hat{p}}[ - \log h(x) ] - \eta^T \EE_{\omega_{\eta^*}}[ \phi(X,Y) ] + A(\eta).
\end{equation}
For distributions with constant base measure $h(x)$ we have that,
$H_{\hat{p}}(\omega_{\eta^*}(x,y)) = H(\omega_{\eta^*}(x,y))$.  By
similar logic for the marginal entropy, and by applying the entropy
chain rule, we have that:
\begin{equation}
  H_{\hat{p}}( \omega_{\eta^*}(x\mid y) ) = H(\omega_{\eta^*}(x,y)) - H(\omega_{\eta^*}(y)).
\end{equation}
The l.h.s.~is the relevant conditional entropy term from the MI
bound~\eqref{eq:varmi_approx}.  The r.h.s.~is the entropy of the joint
and marginal distributions $\omega(\cdot)$ at the optimal parameters,
which is closed form.  We have thus shown that the aforementioned MI
approximation is equivalent to optimizing the variational lower bound
for some cases.


%% \subsection{BACKUP}
%% Consider a pair of joint and marginal distributions in the exponential
%% family,
%% \begin{gather*}
%%   \omega_{\eta}(x,y) = h(x,y)\exp\left\{ \eta^T \phi(x,y) - A(\eta)
%%     \right\}\\ % \equiv \omega_{xy}\\
%%   \omega_{\beta}(y) = h(y)\exp\left\{ \beta^T \phi(y) - A(\beta) \right\}.
%%   %\equiv \omega_y.
%% \end{gather*}
%% When $\omega(y) = \int \omega(x,y) \,\deriv x$ are marginally consistent,
%% we have that the conditional is $\omega(x\mid y) = \omega(x,y) \div
%% \omega(y)$.  As a result, the objective $J(\theta)$ can be
%% re-expressed as,
%% \begin{align}\label{eq:constrained_mibound}
%%   &\min_{\eta, \beta} \;J(\eta,\beta) \equiv \EE_p[ \log \omega_{\beta}(Y) ] - \EE_p[ \log
%%     \omega_{\eta}(X,Y) ] \notag \\
%%   &\;\text{s.t.} \; \EE_{\omega_{\beta}}[ \phi(Y) ] =
%%     \EE_{\omega_{\eta}}[ \phi(Y) ].
%% \end{align}
%% We have assumed here that $\int \omega(x,y) \,\deriv x$ is in the same
%% exponential family as $\omega(y)$.  Then, marginal consistency
%% is equivalent to the moment constraints above.  This assumption does not hold in
%% general but will simplify later discussion.

%% When the marginalization constraints are satisfied we have that
%% $J(\eta,\beta) = J(\theta)$ by construction, where $\theta$ can be
%% expressed in terms of the parameters $\eta$ and $\beta$.  The
%% problem~\eqref{eq:constrained_mibound} is then convex on the
%% constraint set, though not strictly so since many joint and marginal
%% distributions map to the same conditional.  

%% The objective $J(\eta,\beta)$ is not convex off of the constraints.
%% By adding constants $\EE_p[ \log p(X,Y) ]$ and $\EE_p[ - \log p(Y) ]$
%% we have the equivalent expression,
%% \begin{equation*}
%%   J(\eta,\beta) = \text{const.} + \KL{ p_{XY} }{ \omega_{\eta} } - \KL{ p_Y }{
%%     \omega_{\beta} },
%% \end{equation*}
%% which is convex in $\eta$ and concave in $\beta$ by convexity
%% properties of Kullback-Leibler for exponential families.

%% The zero gradient of $J(\eta,\beta)$ yields the moment matching
%% equations, 
%% \begin{align}
%%   \EE_p[ \phi(X,Y) ] &= \EE_{\omega_{\eta}}[ \phi(X,Y)
%%     ] \label{eq:statcond_joint} \\
%%   \EE_p[ \phi(Y) ] &= \EE_{\omega_{\beta}}[ \phi(Y) ]. \label{eq:statcond_marg}
%% \end{align}
%% A solution $\{\eta,\beta\}$ to the above equations is a feasible
%% point, given our assumption that $\int \omega(x,y) \,\deriv x$ and
%% $\omega(y)$ belong to the same exponential family.

%% Consider a model where the marginal $p(y)$ is in the exponential
%% family, for example when $p(x,y)$ is a mixture distribution with
%% discrete label $y$.  In this case, the moment matching condition above
%% means $\omega(y) = p(y)$ and thus,
%% \begin{equation}
%%   \EE_p[ \phi(X,Y) ] = \EE_{p_Y}[ \EE_{\omega_{X\mid Y}}[ \phi(X,y)
%%       \mid Y=y ] ],
%% \end{equation}
%% which is the solution of the unconstrained problem in
%% \SEC\ref{sec:optim}.  As a result, it is also a solution to the
%% constrained problem~\eqref{eq:constrained_mibound}, since the
%% objectives are equal on the constraint set by construction.

%% Given the moment matched solution we have that the cross entropy
%% equals, 
%% \[
%%   \EE_p[ - \log \omega_{\eta} ] = \EE_p[ - \log h(x) ] - \eta^T \EE_p[
%%     \phi(X,Y) ] + A(\eta).
%% \]
%% For distributions with constant base measure $h(x)$ the above equals
%% entropy of $\omega(x,y)$, since $\EE_p[ \phi(X,Y) ] = \EE_\omega[
%%   \phi(X,Y) ]$.  The same holds for the marginal entropy and, thus,
%% the conditional cross entropy, \mbox{$H_p( \omega(X\mid Y) ) = H_\omega(X
%% \mid Y)$}.



